% !TeX root = ../main.tex

\chapter{蒙特卡罗模拟}
\label{chap:Simulation}

氚衰变 $\beta$ 能谱如图 \ref{fig:eSpectrum} 所示。

\begin{figure}[htbp]
	\centering
	\includegraphics[width=0.8\textwidth]{figures/eSpectrum.pdf}
	\caption{氚衰变 $\beta$ 能谱}
	\label{fig:eSpectrum}
\end{figure}

\section{氚衰变 $\beta$ 在不同材料中的射程}

\begin{figure}[htbp]
	\centering
	\includegraphics[width=0.8\textwidth]{figures/AirTrackLength.pdf}
	\caption{氚衰变 $\beta$ 在空气中的射程}
	\label{fig:AirTrackLength}
\end{figure}

\begin{figure}[htbp]
	\centering
	\includegraphics[width=0.8\textwidth]{figures/ArC4H10TrackLength.pdf}
	\caption{氚衰变 $\beta$ 在 97.5\% 氩气和 2.5\% 异丁烷中的射程}
	\label{fig:ArC4H10TrackLength}
\end{figure}

\begin{figure}[htbp]
	\centering
	\includegraphics[width=0.8\textwidth]{figures/PETTrackLength.pdf}
	\caption{氚衰变 $\beta$ 在 PET 材料中的射程($\beta$ 粒子垂直入射 PET 材料,统计径迹的总路程;蓝线表示 $\beta$ 径迹总路程在对应区间内的事例数;红线表示 $\beta$ 粒子走过对应径迹长度时被完全吸收的事例占总事例的比重)}
	\label{fig:PETTrackLength}
\end{figure}

\section{氚衰变 $\beta$ 穿透不同厚度的 PET 材料后在氩气中沉积的能谱}

将氚均匀分布在膜窗的一面上,使其衰变,每次运行发射了 1000000 个氚衰变 $\beta$ 粒子,并记录其穿透 PET 材料后在氩气中沉积的能谱,模型如图 \ref{fig:TritiumBetaPenetratePET} 所示。

\begin{figure}[htbp]
	\centering
	\includegraphics[width=0.5\textwidth]{figures/TritiumBetaPenetratePET.png}
	\caption{氚衰变 $\beta$ 穿透 PET 材料模型}
	\label{fig:TritiumBetaPenetratePET}
\end{figure}

氚衰变 $\beta$ 不经过膜窗,直接在氩气中沉积能量的能谱如图 \ref{fig:FullEdep} 所示。

\begin{figure}[htbp]
	\centering
	\includegraphics[width=0.6\textwidth]{figures/FullEdep.pdf}
	\caption{氚衰变 $\beta$ 在氩气中沉积能量的能谱}
	\label{fig:FullEdep}
\end{figure}

氚在氩气表面衰变时,在氩气中沉积能量的能谱如图 \ref{fig:0Edep} 所示。

氚衰变 $\beta$ 穿透了 \SI{0.1}{\micro\meter} 的 PET 材料,并在灵敏区沉积了能量,如图 \ref{fig:01Edep} 所示。

氚衰变 $\beta$ 穿透了 \SI{0.5}{\micro\meter} 的 PET 材料,并在灵敏区沉积了能量,如图 \ref{fig:05Edep} 所示。

氚衰变 $\beta$ 穿透了 \SI{1}{\micro\meter} 的 PET 材料,并在灵敏区沉积了能量,如图 \ref{fig:10Edep} 所示。

设置不同厚度的 PET 材料,模拟氚衰变 $\beta$ 穿透不同厚度的 PET 材料后在氩气中沉积事例占比,如图 \ref{fig:PETThickness} 所示。

\begin{figure}[htbp]
	\centering
	\begin{subfigure}[b]{0.45\textwidth}
		\includegraphics[width=\textwidth]{figures/0Edep.pdf}
		\caption{}
		\label{fig:0Edep}
	\end{subfigure}
	\hfill
	\begin{subfigure}[b]{0.45\textwidth}
		\includegraphics[width=\textwidth]{figures/01Edep.pdf}
		\caption{}
		\label{fig:01Edep}
	\end{subfigure}

	\begin{subfigure}[b]{0.45\textwidth}
		\includegraphics[width=\textwidth]{figures/05Edep.pdf}
		\caption{}
		\label{fig:05Edep}
	\end{subfigure}
	\hfill
	\begin{subfigure}[b]{0.45\textwidth}
		\includegraphics[width=\textwidth]{figures/10Edep.pdf}
		\caption{}
		\label{fig:10Edep}
	\end{subfigure}
	\caption{氚衰变 $\beta$ 穿透不同厚度的 PET 材料后在氩气中沉积的能谱。 (a) 在氩气表面衰变时,在氩气中沉积的能谱; (b) 穿透了 \SI{0.1}{\micro\meter} 的 PET 材料后在氩气中沉积的能谱; (c) 穿透了 \SI{0.5}{\micro\meter} 的 PET 材料后在氩气中沉积的能谱; (d) 穿透了 \SI{1}{\micro\meter} 的 PET 材料后在氩气中沉积的能谱}
	\label{fig:Edep}
\end{figure}

\begin{figure}[htbp]
	\centering
	\includegraphics[width=0.8\textwidth]{figures/PETThickness.pdf}
	\caption{氚衰变 $\beta$ 穿透不同厚度的 PET 材料后在氩气中沉积事例占比}
	\label{fig:PETThickness}
\end{figure}

\begin{figure}[htbp]
	\centering
	\includegraphics[width=0.8\textwidth]{figures/AirNEdep.pdf}
	\caption{氚衰变 $\beta$ 穿透不同厚度的 空气 后在氩气中沉积事例占比}
	\label{fig:AirThickness}
\end{figure}

\section{氚衰变 $\beta$ 在不同气体室厚度下穿过一层石墨烯材料后在氩气中沉积的能谱}

氚均匀分布在一个圆柱形的空气腔内,空气腔的两端为厚度为0.3nm的石墨烯材料,不同空气腔厚度下,模拟氚衰变 $\beta$ 穿过石墨烯材料后在氩气中沉积的能谱。模型如图 \ref{fig:TritiumBetaPenetrateGraphene} 所示。

\begin{figure}[htbp]
	\centering
	\includegraphics[width=0.6\textwidth]{TritiumBetaPenetrateChamber.png}
	\caption{氚衰变 $\beta$ 在不同气体室厚度下穿过一层石墨烯材料后在氩气中沉积的模型}
	\label{fig:TritiumBetaPenetrateGraphene}
\end{figure}

\begin{figure}[htbp]
	\centering
	\includegraphics[width=0.6\textwidth]{GrapheneEdep.pdf}
	\caption{氚衰变 $\beta$ 在穿过一层 \SI{0.335}{nm} 厚的石墨烯材料后在氩气中沉积的能谱}
	\label{fig:GrapheneEdep}
\end{figure}

\begin{figure}[htbp]
	\centering
	\includegraphics[width=0.6\textwidth]{ChamberNEdep.pdf}
	\caption{氚衰变 $\beta$ 在不同气体室厚度下穿过一层石墨烯材料后在氩气中沉积事例占比}
	\label{fig:ChamberThickness}
\end{figure}

\begin{figure}[htbp]
	\centering
	\includegraphics[width=0.6\textwidth]{ChamberNEdepCut.pdf}
	\caption{氚衰变 $\beta$ 在不同气体室厚度下穿过一层石墨烯材料后可被灵敏区探测到的事例占比}
	\label{fig:ChamberThicknessCut}
\end{figure}
