% !TeX root = ../main.tex

\chapter{Beta Detection Module Outline}

\section{Introduction}
\begin{itemize}
    \item $\upbeta$ 探测的重要性、应用场景、以及应用场景中对于低本底的需求。
    \item 现存的 $\upbeta$ 探测器的方法,以及其优缺点(在低本底方面的不足)。
    \item 本文的目标:设计一个新的 $\upbeta$ 探测器,以满足低本底的需求。
\end{itemize}

\section{Design of the Prototype System}
\subsection{System Architecture}
\begin{itemize}
    \item 系统的整体架构,包括各个部分的功能和作用。
    \item 各个部分之间的连接方式。
\end{itemize}

\subsection{Development of the Prototype TPC}
\begin{itemize}
    \item TPC 的制造原理。
    \item TPC 的详细参数。
    \item TPC 的工作条件。
\end{itemize}

\subsection{Scheme of the Readout Electronics}
\begin{itemize}
    \item 电子学各个部分的功能和作用。
    \item 电子学的详细参数。
\end{itemize}

\subsubsection{Front-end Board (FEB)}
\begin{itemize}
    \item 输入级
    \item CSA 电路
    \item 单端转差分电路
    \item 供电方案
\end{itemize}

\subsubsection{Data Processing Unit (DPU)}
\begin{itemize}
    \item 模拟数字转换器
    \item 供电方案
    \item 时钟方案
\end{itemize}

\subsubsection{FPGA Logic}
\begin{itemize}
    \item 逻辑设计
\end{itemize}

\section{Performance of the Prototype System}

\subsection{Electronics Performance Test}
\begin{itemize}
    \item 电子学基线及噪声测试。
    \item 电子学的线性度及动态范围测试。
\end{itemize}

\subsection{Test with X-ray Source}
\begin{itemize}
    \item X-ray 源的参数和测试条件。
    \item 测试能谱结果。
    \item Micromegas 增益非均匀性修正。
\end{itemize}

\subsection{Test with $\upbeta$ Source}
\begin{itemize}
    \item $\upbeta$ 源的参数和测试条件。
    \item 径迹重建、击中位置分辨率。
    \item 特征提取。
\end{itemize}

\subsection{Test with background radiation}
\begin{itemize}
    \item 本底的测试条件。
    \item 特征提取,与 $\upbeta$ 特征的对比。
\end{itemize}

\subsection{Background Suppression Performance}
\begin{itemize}
    \item 基于决策树的本底抑制方法。
    \item 本底抑制效果。
\end{itemize}

\section{Conclusion}
\begin{itemize}
    \item 本文设计的 $\upbeta$ 探测器的性能。
    \item 本文设计的 $\upbeta$ 探测器的优点。
    \item 未来的工作。
\end{itemize}