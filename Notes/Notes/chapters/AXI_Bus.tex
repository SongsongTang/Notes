\chapter{AXI 总线}
\section{为什么要学习 AXI 总线}
\begin{enumerate}
    \item ZYNQ: 异构芯片,内部总线使用的 AXI 总线。
    \item 纯 FPGA 的 IP 接口也要用 AXI 总线。
    \item 想做高速接口,设计 PCIe、JESD204B、DDR3/4 (APP, AXI), SOC
\end{enumerate}
\section{什么是 AXI 总线}
\paragraph{AXI 总线分类}
\begin{enumerate}
    \item AXI4-Full
    \item AXI4-Lite
    \item AXI4-Stream
\end{enumerate}
AXI: 不是 Xilinx 研发的,是ARM研发的,AMBA 其中的一个,AMBA包括(APB,AHB,AXI)。\par
AXI 总线是一种突发总线,突发传输。一直连续的传输,比如说突发8次传输,就是指传输数据连续的传输8次,不需要每次都发起一次传输请求。\par
\paragraph{AXI 的 5 个通道}
\begin{enumerate}
    \item 写地址通道
    \item 写数据通道
    \item 写响应通道
    \item 读地址通道
    \item 读数据通道
\end{enumerate}
\paragraph{波形解读}
\begin{enumerate}
    \item VALID 和 READY,同时为高代表当前数据有效。
    \item 先写地址,紧接着写数据,最后一个数据跟随 LAST
    \item 先读地址,紧接着读数据,最后一个数据跟随 LAST
    \item 写完数据,有一个写响应,表示写成功。
\end{enumerate}

\section{使用 FPGA 实现 AXI 主接口}
突发传输:1 2 4 8 16 32 64 128 256\par
\subsection{AXI 写实现的步骤}
\begin{enumerate}
    \item 写首地址
    \item 紧接着写首地址,突发传输数据
    \item 控制LAST信号
    \item LAST 脉冲之后,等待一个响应信号
\end{enumerate}
\subsection{AXI 读实现的步骤}
\begin{enumerate}
    \item 写首地址
    \item 紧接着等待从机传输过来的数据,当有效信号和准备信号同时为高时,读出数据
    \item 接收 LAST 信号
\end{enumerate}
\subsection{实现功能:}
\begin{enumerate}
    \item 向从机的0地址写1$\sim$16;
    \item 读从机的0地址;
\end{enumerate}
\section{AXI 传输机制和死锁}
\paragraph{Outstanding address 和 Out-of-order}
\subparagraph{Outstanding address}
AXI 总线的写地址和写数据时分离的,可以在写地址总线上连续发送两次地址,再依次传输写数据。
\subparagraph{Out-of-order}
多对多的情况下,针对不同ID,传输顺利可以任意的,AXI4中只针对从机有效。主机不支持。
\paragraph{什么是死锁}
AXI 总线发生异常,无法正常工作。
\paragraph{为什么会产生死锁}
没有严格遵守 AXI 总线协议
\paragraph{AXI 产生死锁的情况(多对多)}
\begin{enumerate}
    \item 一个主机对多个从机,主机先发送 S3、S2、S1,由于S3 距离非常远,传输非常慢,而S1离得非常近,传输非常快。导致S1先收到了写地址信号,那么S1将会第一个占用M1。但是M1的传输必须是顺序的,第一占用的必须是S3,此时M1死锁。
    \item 多主机对多从机(Out-of-order):两个主机,第一主机发送相同ID到S1、S2.第二个主机发送相同的ID到S2、S1.但是,经过AXI桥,将ID扩展后,对于从机来说是不同ID,可以乱序传输。那么S1可能会先响应第二个主机,那么S2响应第一个主机。主机接收响应乱序,违反AXI4.0协议规范,死锁。
\end{enumerate}